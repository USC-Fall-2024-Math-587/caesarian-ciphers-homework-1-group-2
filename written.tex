\documentclass[12pt]{amsart}
\usepackage{amsmath}
\usepackage{amsthm}
\usepackage{amsfonts}
\usepackage{amssymb}
\usepackage[margin=1in]{geometry}
\usepackage{hyperref}
\hypersetup{
    colorlinks=true,
    linkcolor=blue
}

\theoremstyle{definition}
\newtheorem{theorem}{Theorem}[section]
\newtheorem{lemma}[theorem]{Lemma}
\newtheorem{definition}[theorem]{Definition}
\newtheorem{corollary}[theorem]{Corollary}
\newtheorem{proposition}[theorem]{Proposition}
\newtheorem{conjecture}[theorem]{Conjecture}
\newtheorem{remark}[theorem]{Remark}
\newtheorem{example}[theorem]{Example}
\newtheorem{problem}[theorem]{Problem}
\newtheorem{notation}[theorem]{Notation}
\newtheorem{question}[theorem]{Question}
\newtheorem{caution}[theorem]{Caution}

\begin{document}

\title{Homework 1}

\maketitle

For this week, please answer the following questions from the text. 
I've copied the problem itself below and the question numbers for 
your convenience. 

\begin{enumerate}
	\item (1.2) Decrypt each of the following Caesar encryptions by trying the various 
		possible shifts until you obtain readable text.
		\begin{itemize}
			\item \texttt{LWKLQNWKDWLVKDOOQHYHUVHHDELOOERDUGORYHOBDVDWUHH}
			\item \texttt{UXENRBWXCUXENFQRLQJUCNABFQNWRCJUCNAJCRXWORWMB}
			\item \texttt{BGUTBMBGZTFHNLXMKTIPBMAVAXXLXTEPTRLEXTOXKHHFYHKMAXFHNLX}
		\end{itemize}

	\item (1.3) Use the simple substitution table below
	\begin{center}
		\begin{tabular}{|c |c |c |c |c |c |c |c |c |c |c |c |c |c |c |c |c |c |c |c |c |c |c |c| c| c|}
			\hline
			a & b & c & d & e & f & g & h & i & j & k & l & m & n & o & 
			p & q & r & s & t & u & v & w & x & y & z \\
			\hline
			S & C & J & A & X & U & F & B & Q & K & T & P & R & W & E & 
			Z & H & V & L & I & G & Y & D & N & M & O \\
			\hline
		\end{tabular}
	\end{center}
	\begin{enumerate}
		\item Encrypt the plaintext message
		\begin{center}
			\texttt{The gold is hidden in the garden.}
		\end{center}
		\item Make a decryption table, that is, make a table in which the ciphertext 
			alphabet is in order from A to Z and the plaintext alphabet is mixed up.
		\item Use your decryption table from (b) to decrypt the following message.
		\begin{center}
			\texttt{IBXLX JVXIZ SLLDE VAQLL DEVAU QLB}
		\end{center}
	\end{enumerate}
\item (1.4.c) Each of the following messages has been encrypted using a simple
	substitution cipher. Decrypt them. For your convenience, we have given
	you a frequency table and a list of the most common bigrams that appear
	in the ciphertext. (If you do not want to recopy the ciphertexts by
	hand, they can be downloaded or printed from the web site listed in the
	preface.) In order to make this one a bit more challenging, we have
	removed all occurrences of the word “the” from the plaintext. 

	“A Brilliant Detective”
	\begin{center}
		\ttfamily
		GSZES GNUBE SZGUG SNKGX CSUUE QNZOQ EOVJN VXKNG XGAHS AWSZZ
		BOVUE SIXCQ NQESX NGEUG AHZQA QHNSP CIPQA OIDLV JXGAK CGJCG
		SASUB FVQAV CIAWN VWOVP SNSXV JGPCV NODIX GJQAE VOOXC SXXCG
		OGOVA XGNVU BAVKX QZVQD LVJXQ EXCQO VKCQG AMVAX VWXCG OOBOX
		VZCSO SPPSN VAXUB DVVAX QJQAJ VSUXC SXXCV OVJCS NSJXV NOJQA
		MVBSZ VOOSH VSAWX QHGMV GWVSX CSXXC VBSNV ZVNVN SAWQZ ORVXJ
		CVOQE JCGUW NVA
	\end{center}
		
	The ciphertext contains $313$ letters. Here is a frequency table: 
	\begin{center}
		\ttfamily
		\begin{tabular}{|c||c |c |c |c |c |c |c |c |c |c |c |c |c |c |c |c |c |c |c |c |c |c |c| c| c|}
			\hline
			& V & S & X & G & A & O & Q & C & N & J & U & Z & E & W
			& B & P & I & H & K & D & M & L & R & F \\
			\hline
			Freq  & 39 & 29 & 29 & 22 & 21 & 21 & 20 & 20 & 19 & 13
			      & 11 & 11 & 10 & 8 & 8 & 6 & 5 & 5 & 5 & 4 & 3 &
			2 & 1 & 1 \\
			\hline
		\end{tabular}
	\end{center}

	The most frequent bigrams are: \texttt{XC} (10 times), \texttt{NV} (7
	times), and \texttt{CS}, \texttt{OV}, \texttt{QA}, and \texttt{SX} (6
	times each).	
		
	\item (1.5) Suppose that you have an alphabet of 26 letters.
		\begin{enumerate}
			\item  How many possible simple substitution ciphers are there?
			\item  A letter in the alphabet is said to be fixed if the encryption of the letter is the
				letter itself. How many simple substitution ciphers are there that leave:
			\begin{enumerate}
				\item No letters fixed?
				\item At least one letter fixed?
				\item Exactly one letter fixed?
				\item At least two letters fixed?
			\end{enumerate}
			(Part (b) is quite challenging! You might try doing the problem first with an alphabet 
			of four or five letters to get an idea of what is going on.)
		\end{enumerate}
\end{enumerate}

\end{document}
